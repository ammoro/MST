%% LyX 1.3 created this file.  For more info, see http://www.lyx.org/.
%% Do not edit unless you really know what you are doing.
\documentclass[12pt,english]{article}
\usepackage[T1]{fontenc}
\usepackage[latin1]{inputenc}

\makeatletter

%%%%%%%%%%%%%%%%%%%%%%%%%%%%%% LyX specific LaTeX commands.
%% Bold symbol macro for standard LaTeX users
\newcommand{\boldsymbol}[1]{\mbox{\boldmath $#1$}}

%% Because html converters don't know tabularnewline
\providecommand{\tabularnewline}{\\}

%%%%%%%%%%%%%%%%%%%%%%%%%%%%%% Textclass specific LaTeX commands.
 \newcommand{\lyxaddress}[1]{
   \par {\raggedright #1 
   \vspace{1.4em}
   \noindent\par}
 }

\usepackage{babel}
\makeatother
\begin{document}

\title{MSTamp: A computer code to evaluate the multiple scattering expansion
of the total transition amplitude for nucleon-nucleus scattering}


\author{R. Crespo and A. M. Moro}

\maketitle

\lyxaddress{\begin{center}Departamento de F\'isica, Instituto Superior T\'ecnico,
1049-001 Lisboa, Portugal\end{center}}

Email: raquel@wotan.ist.utl.pt, moro@mary.ist.utl.pt

\tableofcontents{}


\section{Formalism}


\subsection{The nuclear structure}

We describe in this section the scattering of a nucleus assumed to
be well described by a core and two valence particles (C+v$_{1}$+v$_{2}$)
originally in a $|\Psi_{i}\rangle$, state to a final $|\Psi_{f}\rangle$
state, by means of its interaction with a proton, with initial momentum
$\vec{k}_{i}$ and spin $S\sigma$ ($S=1/2$) and final momentum $\vec{k}_{f}$
and spin $S\sigma'$ in the nucleon-nucleus center-of-mass frame.
We describe the initial (final) state with angular momentum of the
valence pair $J^{\pi}(i)$ $\left(J^{\pi}(f)\right)$ and energy $E_{i}$
($E_{f}$) as $\phi_{i}=|J^{\pi}(i),E_{i}\rangle$ $\left(\phi_{f}=|J^{\pi}(f),E_{f}\rangle\right)$,
neglecting the spin of the core. The wave function is then written
as \begin{equation}
\Psi_{n}(\vec{r},\vec{R},R_{c})=\left[\phi_{n}(\vec{r},\vec{R})\otimes\Phi(\vec{\xi}_{c})\right]\end{equation}
where $\Phi(\vec{\xi}_{c})$ is the core internal wave function and
$\phi_{n}(\vec{r},\vec{R})$ the two body wave function of the valence
system relative to the core, for the state n(=i,f),

\begin{equation}
\phi_{n}(\vec{r},\vec{R})=\sum_{\ell\lambda LS}F_{\ell\lambda LSJ_{n}}(r,R)\left[\left[Y_{\ell}(\hat{r})\otimes Y_{\lambda}(\hat{R})\right]_{L}\otimes\left[\chi_{S_{2}}\otimes\chi_{S_{3}}\right]_{S}\right]^{J_{n}M_{n}}\label{eq:wfnn}\end{equation}
In this equation $\lambda$ is the orbital angular momentum between
the core and the center of mass of the valence pair and $\ell$ the
relative angular momentum of the pair. 


\subsection{The transition amplitudes}

We consider then the scattering of a nucleon from a system assumed
to be well described by $\mathcal{N}$ subsystems. The total transition
amplitude T can be written as a multiple scattering expansion in the
transition amplitudes $\hat{t}_{\mathcal{I}}$ for proton scattering
from each projectile sub-system $\mathcal{I}$\cite{Crespo99a  ,Crespo01,Crespo02}

\begin{equation}
T=\sum_{\mathcal{I}}\hat{t}_{\mathcal{I}}+\sum_{\mathcal{I}}\hat{t}_{\mathcal{I}}G_{0}\sum_{\mathcal{J}\neq\mathcal{I}}\hat{t}_{\mathcal{J}}+\cdots\end{equation}
where the propagator $G_{0}=(E^{+}-\mathcal{K})^{(-1)}$, within the
impulse approximation, contains the kinetic energy operator of the
projectile and all the target subsystems. Here E is the kinetic energy
in the overall center of mass frame. In the case of a p scattering
from a 3-body composite system (C+v$_{1}$+v$_{2}$) the single scattering
approximation is

\[
T=T_{C}+T_{v_{1}}+T_{v_{2}}\]



\subsubsection{The single scattering approximation}


\paragraph{Scattering from the valence system}

We assume in here that the valence systems are nucleons. The nucleon-nucleon
scattering can be described by the tensor representation \cite{Crespo02a}.
This is a a convenient general method to express the NN transition
amplitude as a linear combination for the spherical components of
the spin operators of the two interacting particles. In this representation,
the scattering amplitude is written in terms of the tensor of rank
$\kappa$

\begin{eqnarray}
\mathcal{T}_{\kappa q}(a,b)=\sum_{\alpha\beta}(a\alpha b\beta|\kappa q)\tau_{a\alpha}(s_{0})\tau_{b\beta}(s_{1}).\label{bigT}\end{eqnarray}
 as, \begin{eqnarray}
\langle\vec{k}'|t|\vec{k}\rangle=\sum_{\kappa qab}t_{\kappa q}^{(ab)}(\vec{k}',\vec{k})\mathcal{T}_{\kappa q}^{\dagger}(a,b),\end{eqnarray}
 where $\tau_{a\alpha}(s_{0})$ is the irreducible tensor operator
for the projectile particle (0) with spin $s_{0}$ ($a=0,\ldots2s_{0}$);
$\tau_{b\beta}(s_{1})$ is the irreducible tensor operator for the
struck particle (1) with spin $s_{1}$ ($b=0,\ldots2s_{1}$). Explicitly,
since $s_{0}=s_{1}=\frac{1}{2}$, $\tau_{00}({\frac{1}{2}})=1$ and
$\tau_{1\beta}({\frac{1}{2}})=\sigma_{\beta}(1)$, with $\sigma_{\beta}(1)$
the spherical components of $\vec{\sigma}_{1}$ with respect to the
chosen $z$-axis. It follows that the scattering from the valence
can be written as:

\begin{equation}
\langle\vec{k}_{f}S\sigma';\Psi_{n'}|\hat{t}_{[v_{1}]}|\vec{k}_{i}S\sigma;\Psi_{n}\rangle=\sum_{b\beta}\hat{t}_{[b\beta];[v_{1}]}^{\sigma'\sigma}(\omega_{v}\vec{\Delta})\rho_{[b\beta]}^{fi}\left(\frac{m_{v_{2}}}{M_{v}}\vec{\Delta},\frac{M_{C}}{M}\vec{\Delta}\right)\label{eq:sscatv}\end{equation}
where the quantity $\hat{t}_{[b\beta];[v_{1}]}^{\sigma'\sigma}$ is
given in terms of the tensor components of the nucleon-nucleon transition
amplitude as 

\begin{eqnarray*}
\hat{t}_{[b\beta];[v_{1}]}^{\sigma'\sigma} & = & \sum_{\kappa qa\alpha}(-)^{q}\frac{\hat{S}_{v_{2}}}{\hat{b}}\langle S||\tau_{a}(\frac{1}{2};p)||S\rangle\langle S_{v_{1}}||\tau_{b}(\frac{1}{2};v1)||S_{v_{1}}\rangle\\
 & \times & t_{\kappa q}^{(ab)}(\omega\,\vec{\Delta})(S\sigma a\alpha|S\sigma')(a\alpha b\beta|S\sigma')\end{eqnarray*}
The density formfactor is

\begin{eqnarray*}
\rho_{[b\beta]}^{fi}\left(\frac{m_{v_{2}}}{M_{v}}\vec{\Delta},\frac{M_{C}}{M}\vec{\Delta}\right) & = & \sum\hat{\rho}_{[bdc]}\left(\frac{m_{v_{2}}}{M_{v}}\Delta,\frac{M_{C}}{M}\Delta\right)\\
 & \times & \varphi_{2}(b\beta d\delta|c\gamma)(J_{f}M_{f}d\delta|J_{i}M_{i})\sqrt{4\pi}Y_{c\gamma}(\hat{\Delta})\end{eqnarray*}
with

\begin{eqnarray*}
\hat{\rho}_{[bdc]}\left(\frac{m_{v_{2}}}{M_{v}}\Delta,\frac{M_{C}}{M}\Delta\right) & = & \sum\left(i^{\ell}\right)^{*}\left(i^{\ell'}\right)^{*}\varphi_{1}\frac{\hat{\ell}^{2}\hat{\ell}'^{2}\hat{d}^{2}\hat{b}\hat{S}\hat{S}'\hat{L}\hat{L}'\hat{\ell}_{1}\hat{\lambda}\hat{J}_{\tiny{f}}}{\hat{c}}\\
 & \times & (\ell0\ell_{1}0|\ell_{1}'0)(\ell0\ell'0|\mathcal{L}0)(\ell'0\lambda0|\lambda'0)W(SS_{2}S'S_{2};S_{3}b)\\
 & \times & \left\{ \begin{array}{ccc}
b & d & c\\
S' & J_{f} & L'\\
J & J_{i} & L\end{array}\right\} \left\{ \begin{array}{ccc}
\ell & \ell' & c\\
\ell_{1} & \lambda & L\\
\ell_{1}' & \lambda' & L'\end{array}\right\} \\
 & \times & \int r^{2}drR^{2}dRF_{\tiny{\ell_{1}'\lambda'L'S'J_{f}}}(r,R)F_{\tiny{\ell_{1}\lambda LSJ_{i}}}(r,R)\\
 & \times & j_{\ell}\left(\frac{m_{\nu_{2}}}{M_{\nu}}\Delta r\right)j_{\ell'}\left(\frac{M_{C}}{M}\Delta R\right)\end{eqnarray*}
with $\varphi_{1}$and $\varphi_{2}$ phase factors.


\paragraph{Scattering from the core}

We proceed by evaluating the single scattering term from the core,
and assume only a central interaction in this case. The transition
amplitude is then given as

\begin{equation}
\langle\vec{k}_{f}S\sigma';\Psi_{f}|\hat{t}_{[C]}|\vec{k}_{i}S\sigma;\Psi_{i}\rangle=\langle S\sigma';\Phi_{c}|\hat{t}_{[00];[C]}(\omega_{pC}\vec{\Delta})|\Phi_{c};S\sigma\rangle\rho_{[00]}^{fi}\left(0,\frac{M_{v}}{M}\vec{\Delta}\right)\end{equation}
where $M_{v}=m_{v1}+m_{v2}$ and $M=M_{v}+M_{C}$. The form factor
$\rho_{[00]}^{fi}$ is evaluated from the 2-body halo transition density
distribution

\begin{equation}
\rho_{[00]}^{fi}(\vec{\Delta}_{1,}\vec{\Delta}_{2})=\int d\vec{Q}_{1}d\vec{Q}_{2}\phi_{f}(\vec{Q}_{1},\vec{Q}_{2})\phi_{i}(\vec{Q}_{1}+\vec{\Delta}_{1,}\vec{Q}_{2}+\vec{\Delta}_{2}).\label{eq:rho00}\end{equation}
In eq.(\ref{eq:rho00}) $\phi_{n}(\vec{Q}_{1},\vec{Q}_{2})$ is the
fourier transform of the wave function of the twobody valence system
relative to the core. In terms of the wave function, eq.(\ref{eq:wfnn})
we get

\begin{equation}
\rho_{[00]}^{fi}\left(0,\frac{M_{v}}{M}\vec{\Delta}\right)=\hat{\rho}_{[\mathcal{L}]}\left(0,\frac{M_{v}}{M}\vec{\Delta}\right)(J_{f}M_{i}\mathcal{LM}|J_{i}M_{i})\sqrt{4\pi}Y_{\mathcal{LM}}(\hat{\Delta})\end{equation}
with

\begin{eqnarray*}
\hat{\rho}_{[\mathcal{L}]}\left(0,\frac{M_{v}}{M}\vec{\Delta}\right) & = & \sum i^{\mathcal{L}^{*}}[\hat{J}^{f}\widehat{\mathcal{L}}\hat{L}\hat{L}'\varphi]W(LJ_{i}L'J_{f}S\mathcal{L})W(\mathcal{L}L'\ell\;\lambda'L)(\mathcal{L}0\lambda0|\lambda'0)\\
 & \times & \int r^{2}drR^{2}dRF_{\ell\lambda'L'SJ_{f}}(r,R)F_{\ell\lambda LSJ_{i}}j_{\mathcal{L}}\left(\frac{M_{v}}{M}\Delta R\right)\end{eqnarray*}
where $\varphi$ is a phase factor. 


\subsubsection{The reaction observables}

In order to evaluate the scattering observables we write the transition
amplitude in the following way: For the scattering from the core:

\begin{eqnarray*}
\langle\vec{k}_{f}S\sigma';\Psi_{f}|\hat{t}_{[C]}|\vec{k}_{i}S\sigma;\Psi_{i}\rangle & = & \widehat{\mathcal{L}}\left[\langle S\sigma';\Phi_{c}|\hat{t}_{[00];[pC]}(\omega_{C}\vec{\Delta})|\Phi_{c};S\sigma\rangle\hat{\rho}_{[\mathcal{L}]}\left(0,\frac{M_{v}}{M}\vec{\Delta}\right)\right]\\
 & \times & (J_{f}M_{f}\mathcal{LM}|J_{i}M_{i})\mathcal{D}_{\mathcal{M}0}^{\mathcal{L}*}(\hat{\Delta})\end{eqnarray*}
and for the scattering of the valence particle:

\begin{eqnarray*}
\langle\vec{k}_{f}S\sigma';\Psi_{f}|\hat{t}_{[v1]}|\vec{k}_{i}S\sigma;\Psi_{i}\rangle & = & \hat{c}\hat{\rho}_{[bdc]}\left(\frac{m_{v_{2}}}{M_{v}}\Delta,\frac{M_{C}}{M}\Delta\right)\hat{t}_{[b\beta];[v]}^{\sigma'\sigma}\\
 & \times & (b\beta d\delta|c\gamma)(J_{f}M_{f}d\delta|J_{i}M_{i})\mathcal{D}_{\gamma0}^{c*}(\hat{\Delta})\end{eqnarray*}


For each final state the differential cross section can be readily
evaluated from these expressions by calculating the square and cross
products.


\section{Namelist description}


\subsection*{MST namelist:}

\begin{itemize}
\item \textbf{qmax: }
\item \textbf{tlab}: projectile energy in laboratory frame.
\item \textbf{thmin, thmax, dth}: angular range and step for calculated
cross sections.
\item \textbf{dry: }\\
\textbf{dry=F}: Skip reading 3B wf \textbf{}
\end{itemize}

\subsection*{QUAD1 namelist:}

\begin{itemize}
\item \textbf{qmaxr}: 
\item \textbf{quin}: 
\item \textbf{mquadi}: 
\item \textbf{mquado:}
\end{itemize}

\subsection*{QUAD2 namelist:}

\begin{itemize}
\item \textbf{qmaxrd}:
\item \textbf{quind}: 
\item \textbf{mquadid:}
\item \textbf{mquadod}: 
\end{itemize}

\subsection*{QUAD3 namelist:}

\begin{itemize}
\item \textbf{rmaxr} : 
\item \textbf{rin} :
\item \textbf{mrquadi} : 
\item \textbf{mrquado} : 
\end{itemize}

\subsection*{PROJ namelist:}

\begin{itemize}
\item \textbf{massp}: projectile mass
\item \textbf{zp}: projectile charge
\item \textbf{jp}: spin of projectile
\end{itemize}

\subsection*{TARG namelist:}

\begin{itemize}
\item \textbf{masst}: target mass
\item \textbf{zt}: target charge 
\item \textbf{ncl}: number of clusters.
\item \textbf{inelcb}: ?????
\item \textbf{nustates}: total number of states.
\item \textbf{quais(1:nustates-1)}: vector array to select those $J^{\pi}$
(inelastic) components of the wavefunction that will be taken into
account in the calculations. $quais(i)\neq0$ means that component
$i$ will be included.
\item \textbf{irho}: ???
\end{itemize}

\subsection*{KAPAS namelist:}

\begin{itemize}
\item \textbf{k0000, k1100, k0111, k1120, k1121, k1122}: If \emph{kabkq}=1
the tensor component $t_{\kappa q}^{(ab)}(\omega\,\vec{\Delta})$
of the NN amplitude will be taken into account.
\end{itemize}

\subsection*{TCLUS namelist: }

\begin{itemize}
\item \textbf{mtclus}: mass of this cluster (u.m.a.)
\item \textbf{spin}: cluster intrinsic spin
\item \textbf{ztclus}: cluster charge
\item \textbf{ttype}: Type of T-matrix\\
0=Call MSO program to calculate T-matrix\\
1=Read S-matrix from fortran unit 110, using the format: \emph{sreal,
simag, l, j}
\end{itemize}

\section{Input/Output files}

\begin{tabular}{|l|l|c|c|}
\hline 
\textbf{Unit}&
\textbf{Description}&
\textbf{Format}&
\textbf{Printed with}\tabularnewline
\hline
\hline 
\multicolumn{4}{|c|}{\textbf{NNAMP}}\tabularnewline
\hline 
10 (nnamp.in)&
&
&
\tabularnewline
\hline 
22 / 220&
Off-shell/on-shell isoscalar NN amplitudes&
&
aeoff/aeon=1\tabularnewline
\hline 
23 / 230&
Off-shell/on-shell isovector NN amplitudes&
&
aeoff/aeon=1\tabularnewline
\hline 
32 / 320&
Off-shell/on-shell pn NN amplitudes&
&
aeoff/aeon=3\tabularnewline
\hline 
33 / 330&
Off-shell/on-shell T=0&
&
aeoff/aeon=2\tabularnewline
\hline 
34 / 340&
Off-shell/on-shell T=1&
&
aeoff/aeon=2\tabularnewline
\hline 
&
&
&
\tabularnewline
\hline 
\multicolumn{4}{|c|}{\textbf{MSO}}\tabularnewline
\hline 
&
&
&
\tabularnewline
\hline 
&
&
&
\tabularnewline
\hline 
&
&
&
\tabularnewline
\hline 
14&
Input configuration densities&
&
\tabularnewline
\hline 
11->smat.out&
S-matrix (output)&
&
\tabularnewline
\hline 
110&
S-matrix (input)&
&
\tabularnewline
\hline 
25->xsec.out&
Elastic cross section angular distribution&
&
\tabularnewline
\hline 
\multicolumn{4}{|c|}{\textbf{MST}}\tabularnewline
\hline 
mst.in&
Input parameters&
&
\tabularnewline
\hline 
&
&
&
\tabularnewline
\hline
7&
Input wavefunction&
&
\tabularnewline
\hline
20&
c.m. transition densities&
&
\tabularnewline
\hline
21&
Valence transition densities&
&
\tabularnewline
\hline
75&
Core on-shell T-matrix&
&
\tabularnewline
\hline
70&
Total and core cross section&
$\theta$~$\sigma_{tot}$($\theta$) ~$\sigma_{core}$($\theta$)~&
\tabularnewline
\hline
65&
Double cross-sections&
&
\tabularnewline
\hline
&
&
&
\tabularnewline
\hline
\end{tabular}


\section{Test examples}

\bibliographystyle{unsrt}
\bibliography{referRC}

\end{document}
