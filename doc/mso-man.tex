%% LyX 1.1 created this file.  For more info, see http://www.lyx.org/.
%% Do not edit unless you really know what you are doing.
\documentclass[12pt,english]{article}
\usepackage[T1]{fontenc}
\usepackage[latin1]{inputenc}
\usepackage{babel}
\usepackage{verbatim}

\makeatletter

%%%%%%%%%%%%%%%%%%%%%%%%%%%%%% LyX specific LaTeX commands.
\providecommand{\LyX}{L\kern-.1667em\lower.25em\hbox{Y}\kern-.125emX\@}

%%%%%%%%%%%%%%%%%%%%%%%%%%%%%% Textclass specific LaTeX commands.
 \newcommand{\lyxaddress}[1]{
   \par {\raggedright #1 
   \vspace{1.4em}
   \noindent\par}
 }

\makeatother
\begin{document}

\title{MSOamp: A computer code to evaluate the multiple scattering expansion
of the optical potential for nucleon-nucleus scattering}


\author{R. Crespo and A. M. Moro}

\maketitle

\lyxaddress{\centering Departamento de F\'isica, Instituto Superior T\'ecnico,
1049-001 Lisboa, Portugal}

Email: raquel@wotan.ist.utl.pt, moro@mary.ist.utl.pt

\tableofcontents{}


\section{Introduction}

MSOamp is a computer code to evaluate the elastic scattering observables
for proton-nucleus elastic scatering from an optical potential in
the momentum space configuration. The optical potential is constructed
as a multiple scattering expansion in terms of the free NN transition
amplitude and the target densities\cite{Ker59,Wat53,Crespo92}. The
free NN transition amplitude is calculated in the momentum space configuration
from a realistic NN interaction such as Paris or Bonn \cite{Lacombe80,Machleidt87}. 

The code is constructed in fortran 90. The inputs are given through
a user friendly namelist.

Details of the formalism and some applications can be found in \cite{Crespo02b}.


\section{The formalism}


\subsection{The Lippmann-Schwinger equation: method of solution}

The Lippmann-Schwinger equation for the elastic scattering problem,
for the potential U is \begin{equation}
T'=U+UG_{0}T'
\end{equation}
with the Green's function \( G_{0}=(E-K_{0}+i\epsilon )^{-1} \)where
\( K_{0} \) is the projectile kinetic energy operator. In order to
solve this equation, we introduce the partial wave equation \begin{equation}
|\vec{k}\rangle =\sqrt{\frac{2}{\pi }}\sum _{JLSM}i^{L}|k(LS)JM\rangle Y_{JL}^{M+}(\hat{k}),
\end{equation}
leading to \begin{equation}
\label{eq:TL}
T'_{L\pm }(k,k_{0})=U_{L\pm }(k,k_{0})+\frac{2}{\pi }\int ^{\infty }_{0}dpp^{2}U_{L\pm }(k,p)G_{0}(p)T'_{L\pm }(p,k_{0})
\end{equation}
where \( G_{0}(p) \) is the Gree's function in the momentum representation
\begin{equation}
G_{0}(p)=\frac{2\mu _{NA}}{\hbar ^{2}}\frac{1}{k^{2}_{0}-p^{2}+i\epsilon }
\end{equation}
 and  \( L_{\pm }=L\pm 1/2 \) denotes the eigenvalues of the total
angular momentum J. It is more convenient to work with the principal
value form of the Green's function, 

\begin{equation}
G_{0}(p)=G_{0}^{P}(p)-i\pi \frac{2\mu _{NA}}{\hbar ^{2}}\frac{\delta (k_{0}-p)}{k_{0}}
\end{equation}
where \( G_{0}^{P}(p) \) indicates the principal value. Substituting
this eq into the partial wave expansion of the transition amplitude
eq. (\ref{eq:TL}) we get\begin{equation}
R_{L\pm }(k,k_{0})=U_{L\pm }(k,k_{0})+\frac{2}{\pi }\int ^{\infty }_{0}dpp^{2}U_{L\pm }(k,p)G^{p}_{0}(p)R_{L\pm }(p,k_{0})
\end{equation}
where the partial-wave R-matrix is related to the partial-wave T matrix
by the Heitler equation\begin{equation}
\label{eq:RL}
R_{L\pm }(k,k_{0})=T'_{L\pm }(k,k_{0})\left[ 1-i\frac{2\mu _{NA}}{\hbar ^{2}}k_{0}T'_{L\pm }(k_{0},k_{0})\right] ^{-1}
\end{equation}
From this equation it follows immediately that the on-shell T-matrix
\( T'_{L\pm }(k_{0},k_{0}) \) is given by\begin{equation}
\label{eq:TL-RL}
T'_{L\pm }(k_{0},k_{0})=\frac{R_{L\pm }(k_{0},k_{0})}{1+i\rho _{\epsilon }R_{L\pm }(k_{0},k_{0})}
\end{equation}
with \( \rho _{\epsilon }=2\mu _{NA}/\hbar ^{2} \). In the KMT multiple
scattering formalism, the transition amplitude for elastic scattering,
T, is related to the transition amplitude associated with the potential
U, through the relation \( T=\frac{A}{A-1}T' \). Defining the normalized
T-matrix and R-matrix according to \( \hat{T}_{L\pm }(k,k')=-\rho _{\epsilon }T_{L\pm }(k,k') \)
and \( \hat{R}_{L\pm }(k,k')=-\rho _{\epsilon }R_{L\pm }(k,k') \)
respectively we obtain \begin{equation}
\hat{T}_{L\pm }(k_{0},k_{0})=\frac{\hat{R}_{L\pm }(k_{0},k_{0})}{1-i\rho _{\epsilon }\hat{R}_{L\pm }(k_{0},k_{0})}
\end{equation}
and thus from eq. (\ref{eq:RL}) and (\ref{eq:TL-RL}) the normalized
half-off shell T matrix \( \hat{T}_{L\pm }(k,k_{0}) \), is given
by\begin{equation}
\hat{T}_{L\pm }(k,k_{0})=\hat{T}_{L\pm }(k,k_{0})\left[ \frac{A}{A-1}+i\hat{T}_{L\pm }(k,k_{0})\right] 
\end{equation}
These normalized on-shell transition matrix elements are related to
the nuclear phase shifts according to\begin{equation}
\hat{T}_{L\pm }(k_{0},k_{0})=\frac{\exp (2i\delta _{L\pm })-1}{2i}=\exp (i\delta _{L\pm })sin\delta _{L\pm }
\end{equation}
and the reflection coefficients are \( \eta _{L\pm }=|\exp (2i\delta _{L\pm })|=|2i\hat{T}_{L\pm }(k_{0},k_{0})+1| \).
We shall omit, from now on, the partial wave labels in order to simplify
the notation. 

The numerical solution of the partial wave form of the LS equation,
eq.(\ref{eq:TL}), involves the evaluation of the principal value
of the integral together with a discretization and truncation of the
range of the momentum variable p. We now briefly describe the method
to solve the LS equation \cite{Landau82,Crespo92}.

The principal value is calculated using the Haftel-Tabakin method
. This consists of subtracting a smooth integrand whose contribution
to the integral is zero, i.e.\begin{equation}
\label{eq:R-Haftel-Tabakin}
R(k,k_{0})=U(k,k_{0})+\frac{2}{\pi }\frac{2\mu _{NA}}{\hbar ^{2}}\int ^{\infty }_{0}dp\left[ \frac{p^{2}U(k,p)R(p,k_{0})-k^{2}_{0}U(k,k_{0})R(k_{0},k_{0})}{k^{2}_{0}-p^{2}}\right] 
\end{equation}
The discretization of the momentum integral is achieved by introducing
N Gaussian quadrature points \( (k_{j};j=1,N) \) each weighted by
\( w_{i} \). All N integration points, are required not to be equal
to \( k_{0} \). The discretized version of the previous equation,
eq.(\ref{eq:R-Haftel-Tabakin}) reduces to \begin{eqnarray*}
R(k_{i},k_{0})= & U(k_{i},k_{0})+\frac{2}{\pi }\sum _{j=1}^{N}\frac{2\mu _{NA}}{\hbar ^{2}}w_{j}k_{j}^{2}\frac{U(k_{i},k_{j})R(k_{j},k_{0})}{k^{2}_{0}-k^{2}_{j}} & \nonumber \\
 & -\frac{2}{\pi }\sum ^{N}_{j=1}\frac{2\mu _{NA}}{\hbar ^{2}}k^{2}_{0}\frac{w_{j}}{k^{2}_{0}-k^{2}_{j}}U(k_{i},k_{0})R(k_{0},k_{0}) & \nonumber 
\end{eqnarray*}
Calling \( k_{N+1}=k_{0} \), we obtain the following system of N+1
equations:\begin{equation}
\sum ^{N+1}_{j=1}F(k_{i},k_{j})R(k_{j},k_{m})=U(k_{i},k_{m})
\end{equation}
or simbolically \begin{equation}
[F]_{(N+1)x(N+1)}[R]_{N+1}=[U]_{N+1}
\end{equation}
with\begin{equation}
F(k_{i},k_{j})=\delta _{ij}+U(k_{i},k_{j})\hat{W}_{j}
\end{equation}
and\begin{equation}
\hat{W}_{j}=\frac{2}{\pi }\frac{2\mu _{NA}}{\hbar ^{2}}k^{2}_{j}\frac{w_{j}}{k^{2}_{j}-k^{2}_{0}};j\leq N
\end{equation}


\begin{equation}
\hat{W}_{j}=-\frac{2}{\pi }\frac{2\mu _{NA}}{\hbar ^{2}}k^{2}_{0}\sum ^{N}_{m=1}\frac{w_{m}}{k_{m}^{2}-k^{2}_{0}};j=N+1
\end{equation}
The coulomb interaction is included in an approximate way as described
in \cite[and references therein]{Crespo90}


\subsection{The optical potential}

We assume that the target nucleus of number of mass A is well described
by N clusters. Within the Multiple Scattering expansion of the Optical
potential such as formalated by KMT\cite{Ker59,Crespo92}, the optical
potential can be written as an expantion in terms of the free NN transition
amplitude. Within the single scattering \( t\rho  \) approximation,
the matrix elements of the potential are in the momentum space configuration:
\begin{equation}
\langle \vec{k}'|U|\vec{k}\rangle =\frac{A-1}{A}\sum \rho ^{i}_{p}(q)\bar{t}_{pp}(\omega ,q,Q/2,\phi )+\rho ^{i}_{n}\bar{t}_{nn}(\omega ,q,Q/2,\phi )
\end{equation}
where \( \rho ^{i}_{n} \)and \( \rho ^{i}_{p} \)are the nuclear
matter density distributions for the protons and neutrons respectively
for each cluster. Here, \( \bar{t}_{pp} \) and \( \bar{t}_{nn} \)
are the spin averaged proton-proton and neutron-neutron respectively
amplitudes evaluated at the appropriate energy \( \omega  \), \( \phi  \)
is the angle between the vectors \( \vec{Q}=(\vec{k}'+\vec{k})/2 \)
and \( \vec{q}=(\vec{k}'-\vec{k}) \) . In this eq. other spin components
of the NN transition amplitude for the case of proton elastic scattering
from non zero spin targets are not included. In this case, the potential
can only have a central and a spin-orbit term\begin{equation}
\langle \vec{k}'|U|\vec{k}\rangle =\langle \vec{k}'|U^{c}|\vec{k}\rangle +i\vec{\sigma }\cdot \vec{n}\langle \vec{k}'|U^{ls}|\vec{k}\rangle 
\end{equation}
with \( \sigma  \) the spin operator for the projectile and \( \vec{n}=\vec{\kappa }\times \vec{\kappa }' \).
According to the last paragraph we need to know the partial wave decomposition
of this potential \( U_{LJ}(k',k) \). It follows that, this can be
written as:\begin{equation}
\label{eq:ULJ}
U_{LJ}(k',k)=U_{L}^{c}(k',k)+C_{LJ}U_{L}^{ls}(k',k)
\end{equation}
with the partial wave for the central component given by \begin{equation}
U_{L}^{c}(k',k)=\pi ^{2}\int ^{+1}_{-1}d(cos\theta _{kk'})P_{L}(cos\theta _{kk'})U^{c}(\vec{k}',\vec{k})
\end{equation}
and for the spin-orbit \begin{equation}
U_{L}^{ls}(k',k)=\frac{\pi ^{2}}{L(L+1)}\int ^{+1}_{-1}d(cos\theta _{kk'})sin^{2}\theta _{kk'}P^{1}_{L}(cos\theta _{kk'})U^{ls}(\vec{k}',\vec{k})
\end{equation}
The geometric coefficients in eq.(\ref{eq:ULJ}) are the matrix elements
of the spin-orbit operator\begin{equation}
C_{LJ}=\langle JL|\vec{L}\cdot \vec{\sigma }|JL\rangle =J(J+1)-L(L+1)-3/4
\end{equation}



\section{About the code}


\subsection{Source files}


\subsection{Source files and compilation}

The source files of the MSOamp program are organized into two directories,
\emph{nnsrc} and \emph{msosrc}. The former contains the NNamp source
files, while the latter contains those files specific of the MSO formalism.
Each one of these directories contains a Makefile. There exits also
a Makefile in the top directory of the distribution. If needed, this
can be edited and customized before compilation. Notice that the compiler
name and options are included by means of {*}.def files. Presently,
two of these files are included: \emph{fujitsu.def} and \emph{pgf90.def}.
They contain the compilation specifications for the Fujitsu and PGF90
compilers, respectively. If you have a different f90 compiler, you
may want to create your own {*}.def file, with the appropriate options. 


\subsection{Compilation}

Once the top Makefile have been customized, the NNamp program can
be compiled with:

\begin{verbatim}prompt> cd nnsrc ; make\end{verbatim}

Similarly, to compile the MSOamp program:

\begin{verbatim}prompt> cd msosrc; make\end{verbatim}

Alternatively, from the top directory to can compile the NNamp program
typing:

\begin{verbatim}prompt> make nnamp \end{verbatim} 

\noindent and the MSOamp program with:

\begin{verbatim}prompt> make msoamp \end{verbatim} 

Notice that, whenever the MSO program is compiled, the NNamp program
will also be compiled. 

After compilation, you can delete the {*}.o files by typing: 

\begin{verbatim}prompt> make clean \end{verbatim}

To delete {*}.o as well as MSOamp / NNamp output files: 

\begin{verbatim}prompt> make superclean \end{verbatim}

To install the executables in the compilation (mso and nnamp) into
the destination specified at the top Makefile, type:

\begin{verbatim}prompt> make install \end{verbatim}


\subsection{Compilation}


\section{Description of namelists input variables}


\subsection*{MSO namelist: }

Genearal options:

\begin{itemize}
\item \textbf{tlab}: relative kinetic energy projectile-target in laboratory
system.
\item \textbf{offshell}: T/F
\item \textbf{kmt}: KMT (kmt=T) or Watson (kmt=F) potential. The former
contains the multiplying factor at/(at-1), with at the target atomic
number.
\item \textbf{igwf}: If T, calculates wavefunction up to lstore.
\item \textbf{lstore}: max. partial wave for wf.
\item \textbf{nr}: if nr is not 0, a relativistic correction is performed. 
\item \textbf{ucnr, ucni}: central renormalization constants.
\item \textbf{usnr, usni}: s.o. renormalization constants.
\end{itemize}

\subsection*{COUL namelist:}

\begin{itemize}
\item \textbf{coulomb}: Coulomb correction\\
=0: no coulomb\\
=1: coulomb amplitude and phase change included~\\
=2: coulomb amplitude, but not phase change\\
=3: realistic (HO shape) charge distribuion\\
=3: uniform charge sphere distribution
\item \textbf{rcut} :cut-off radius for subtracted momentum space method
\cite{Crespo90}.
\item \textbf{alfa, beta, gama}: parameters for the realistic charge distribution
of the form:\begin{equation}
\rho (q)=\left\{ 1-\gamma ^{2}q^{2}-\alpha q^{4}\right\} e^{-\beta q^{2}}
\end{equation}

\end{itemize}

\subsection*{HIGHORDER namelist:}

Specifications for high order calculations (second order onward)

\begin{itemize}
\item \textbf{second}:\\
= 1 vopt1\\
= 2 vopt1 + vopt2(local)\\
= 3 vopt1 + vopt2 (non local) \\
= 4 vopt1 + vopt2 (full nonlocal)
\item \textbf{isflip}:
\item \textbf{ich}: 
\item \textbf{ion2}: 
\item \textbf{lamx}:
\item \textbf{lbmx}: 
\item \textbf{l1mx}:
\item \textbf{lfmx}:
\item \textbf{rmaxr}: maximum radius
\item \textbf{quin}: maximum radius for inner region
\item \textbf{mquadi}: number of inner quad points (multiple of 6)
\item \textbf{mquado}: number of outer quad points (multiple of 6)
\end{itemize}

\subsection*{PROJ namelist:}

Projectile properties. Presently, the program only supports structureless
projectiles. Thus, unlike the target, no clusters can be defined.

\begin{itemize}
\item \textbf{massp}: mass number (in a.m.u.)
\item \textbf{zp}: charge
\end{itemize}

\subsection*{TARG namelist:}

Target properties

\begin{itemize}
\item \textbf{masst}: mass number (in a.m.u.)
\item \textbf{zt}: charge
\item \textbf{at}: number of nucleons
\item \textbf{ncluster}: number of clusters
\end{itemize}

\subsection*{CLUSTER namelist:}

After the \textbf{targ} namelist, \emph{ncluster} namelists will be
read. Each one contains the information required to construct the
density for one the clusters. Notice that the different cluster do
not neccesary have to use the same model density.

\begin{itemize}
\item \textbf{type}, \textbf{shape}: specify the kind of density. External
densities (type=1, shape=5) are read from fortran unit 4. The first
line contains the number of points (\emph{nramax}) and the step (\emph{drx}),
in free format. Then, nramax density points will be read, assuming
that the first one corresponds to the origin. \\
\begin{tabular}{|cc|cl|}
\hline 
\multicolumn{2}{|c|}{type}&
\multicolumn{2}{c|}{shape}\\
\hline
\hline 
0&
Numerical potential&
0&
Woods-Saxon\\
&
&
1&
Gauss\\
&
&
2&
Yukawa\\
&
&
3&
Hulthen\\
&
&
4&
cosh\\
&
&
5&
External\\
\hline 
1&
Analytical potential&
0&
Harmonic-oscillator s.p.\\
&
&
1&
Three-parameter Fermi\\
&
&
2&
G3 distribution\\
\hline
\end{tabular}
\item \textbf{nzclus}: number of protons within this cluster.
\item \textbf{nnclus}: number of neutrons within this cluster.
\end{itemize}

\subsection*{BSPARM namelist:}

For type=0, shape=0-4 the bsparm namelist is then read.

\begin{itemize}
\item \textbf{cmass, vmass}: core and bound particle masses.
\item \textbf{zc, zv}: core and bound particle charges.
\item \textbf{nramax}: number of integration steps.
\item \textbf{drx}: radial step for numerical integration.
\item \textbf{dmat}:fine tuning of interior-exterior matching. Usually input
0. {*} If no state found try increase of dmat to of order 10.
\item \textbf{nshell}: number of shells.
\end{itemize}

\subsection*{BSHELL namelist:}

Parameters for this specific shell.

\begin{itemize}
\item \textbf{bengy}: binding energy.
\item \textbf{vdepth}, \textbf{wr0, wal}: potential depth, radius and diffuseness.
\item \textbf{wls}: spin-orbit strength.
\item \textbf{is2}=2{*}valence particle spin \( s \).
\item \textbf{lmoma}= orbital angular momentum \( \ell  \).
\item \textbf{j2a} : 2{*}j, where \( j=\ell +s \).
\item \textbf{nodd}=number of nodes.
\item \textbf{norba}=occupation number of this shell.
\end{itemize}

\subsection*{BS3PF namelist:}

If type=1, shape=5 density is defined, a \textbf{\&bs3pf/} namelist
is then read. In configuration space, it obeys to the form:\begin{equation}
\label{Eq:rho-fermi}
\rho (r)=\rho _{0}\frac{1+w(r/c)^{2}}{\exp [(r-c)/z]}
\end{equation}
where \( \rho _{0} \) is a normalization constant and \( w,\, c,\, z \)
are free parameters. 

\begin{itemize}
\item \textbf{w3p, z3p}, \textbf{c3p}: Fermi parameters for protons.
\item \textbf{w3n, z3n, c3n}: Fermi parameters for neutrons.
\item \textbf{nramax}: number of radial points to calculate the density.
\item \textbf{drx}: step size (fm).
\end{itemize}

\subsection*{HOPARM namelist:}

For type=1, shape=0 densities, harmonic oscillator parameters are
read. First, a \textbf{hoparm} namelist is introduced, which currently
only contains the number of shells that will be next read by means
of \textbf{hoshell} namelists.

\begin{itemize}
\item \textbf{hoparm}: number of harmonic-oscillator shells to be read.
\end{itemize}

\subsection*{HOSHELL namelist: }

Provides the HO parameters and occupations numbers for each shell.

\begin{itemize}
\item \textbf{aap, aan}: HO parameters for protons and neutrons, respectively.
\item \textbf{zorba, norba}: occupation number (protons and neutrons) for
this shell.
\end{itemize}

\section{Test examples}


\subsection{p+\protect\( ^{16}\protect \)O with HO formactors}

In describing \( ^{16} \)O we consider a shell model harmonic oscillator
(HO) description where the center-of-mass corrections are neglected.
Thus, the nucleus matter density distribution is directly evaluated
from the HO single particle wave functions using the relative frequency
obtained by Donnelly and Walker \cite{Donnelly69}. 

\verbatiminput{examples/mso_o16_ho.in}


\subsection{n+\protect\( ^{16}\protect \)Pb with 3-parameter Fermi formactor}

In this example, the density distribution of the \( ^{208} \)Pb nucleus
has been modeled with the three-parameter Fermi form, Eq.\ (\ref{Eq:rho-fermi}).
The parameters where taken from \cite{Barrett79}.

\verbatiminput{examples/mso_pb_g3.in}


\subsection{p+\protect\( ^{9}\protect \)Li with external (Hartree-Fock) formfactors }

In describing the \( ^{9} \)Li ground state, we use a shell model
Hartree-Fock description as described in the work of \cite{Sagawa92}.
This is read externally from fortran unit file 4. The first line of
this line contains the number of points and radial step.

\verbatiminput{examples/mso_li9_hf.in}


\subsection{p+\protect\( ^{9}\protect \)Li with mixed (cluster) formfactors }

As a check of of the dependence of the scattering observables on the
model adopted for the matter density, we repeat in this example the
calculation for p+\( ^{9} \)Li, with a different model for the \( ^{9} \)Li.
Now we assume a cluster model for \( ^{9} \)Li as in \cite{Crespo96}
consisting of an \( \alpha  \)-particle-like core, four neutrons
in the \( p_{3/2} \) shell coupled to spin zero, and a proton in
the \( p_{3/2} \) shell. We take a Gaussian distribution for the
nucleons in the alpha core. For the other nucleons we consider two
cases: an harmonic oscilator model (here called Cluster I) where the
valence neutrons and protons are described within the harmonic oscillator
single particle model. The ranges of the parameters correspond to
the parameter set (a) found in \cite{Crespo96}. We also take the
valence nucleons as bound to the core by a Woods-Saxon potential with
parameters adjusted to obtain the valence neutron and proton separation
energy (here called cluster II).

\verbatiminput{examples/mso_li9_clus.in}

Note that all these examples require, together with the MSOamp input
file (typically mso.in), the NNamp input file. In these examples we
have used:

\verbatiminput{examples/nnamp.in}

\bibliographystyle{unsrt}
\bibliography{referRC}

\end{document}
